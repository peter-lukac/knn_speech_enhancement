% Created 2020-05-08 Fri 23:40
% Intended LaTeX compiler: pdflatex
\documentclass[11pt]{article}
\usepackage[utf8x]{inputenc}
\usepackage[T1]{fontenc}
\usepackage{graphicx}
\usepackage{grffile}
\usepackage{longtable}
\usepackage{wrapfig}
\usepackage{rotating}
\usepackage[normalem]{ulem}
\usepackage{amsmath}
\usepackage{textcomp}
\usepackage{amssymb}
\usepackage{capt-of}
\usepackage{hyperref}
\author{Peter Lukač, Jakub Zárybnický}
\date{\today}
\title{Speech Enhancement}
\hypersetup{
 pdfauthor={Peter Lukač, Jakub Zárybnický},
 pdftitle={Speech Enhancement},
 pdfkeywords={},
 pdfsubject={},
 pdfcreator={Emacs 26.3 (Org mode 9.1.9)}, 
 pdflang={English}}
\begin{document}

\maketitle

\section{Introduction}
\label{sec:orgeca9cd3}
Speech enhancement is one of the tasks in the domain of audio processing, others
being source separation, speech recognition, or source localization. In
particular we focus on the task of single source speech enhancement. In this
report, we present our research into existing methods of speech enhancement, our
approach, and compare the results we obtained with results of other attempts.

\section{Task}
\label{sec:org8c8b9f0}
Given a noisy sound signal, the processing pipeline aims to enhance the contrast
of the speech signal against noise, with the ideal outcome being a clean signal
without any noise.

There is a set of standard metrics for evaluating the quality of de-noising:
\begin{itemize}
\item Signal-to-Noise Ratio (SNR) is the ratio of the clean signal to noise in the
estimated signal in the signal, quantified by the average power.

\item Signal-to-Distortion Ratio (SDR) enhances SNR to account for distortion as well.

\item Short-term Objective Intelligibility (STOI) measures uses time-frequency band
analysis over short windows (\textasciitilde{}400ms) comparing the clean and noisy signals.

\item Perceptual Evaluation of Speech Quality (PESQ) is the standard method of
evaluating speech enhancement techniques. On a scale of 1 (terrible) to 5
(great), this is a composite measure of quality.
\end{itemize}

\section{Existing solutions}
\label{sec:orgfc9962a}
Speech enhancement is a rather traditional audio processing task, commonly
performed in ways other than neural networks. Usual approaches include various
filtering techniques (Wiener, subspace filtering, spectral subtraction), or
spectral restoration using various objective estimators.

However, these approaches often improve only speech quality and not
intelligibility, and often also introduce new distortions into the synthesized
signal. This is also reflected in commonly metrics (the above-presented ones but
also others), where it is the time-frequency spectrum that is compared between
clean and noisy signals.

Neural network-based approaches usually rely on a base of convolutional layers,
in combination with a fully-connected or recurrent layers as well. Most recent
approaches use recurrent layers - bi-directional LSTM in particular, which is
what we have used in the end.

Unfortunately, there have been no comprehensive review articles focusing on
speech enhancement using neural networks since 1999 \cite{wan1999networks},
despite the number of review articles focusing on speech recognition using a
variety of approaches (the most recent one being a review
\cite{nassif2019speech} from 2019).

In evaluating out network's performance, we've used a small-scale review
presented as part of a paper presenting a combination of MSE estimators with a
convolutional LSTM network (DeepXi \cite{NICOLSON201944}), and in particular the
PESQ measure.

\section{Solution}
\label{sec:org0228d4a}
Our solution uses TensorFlow as its basis (the Keras interface in particular),
as well as a number of sound processing libraries for preprocessing the raw
audio files (namely Librosa and SoundFile).

The neural network we use is composed of four convolutional layers, two LSTM
layers, and finally four deconvolutional layers. While our experiments have
included fully-connected, bidirectional LSTM, or other layers, this combination
is one that has produced the best results.

The dataset we have used is a combination of the OpenSLR LibriSpeech sample
library \cite{7178964}, and a number of noise sample sets (cars, cafés, noisy
street, heavy machinery, \ldots{}). The network has been trained using the Adam
optimiser and MSE loss, in 64 epochs of minibatch.

\section{Evaluation}
\label{sec:org07a67a4}
In obtaining the PESQ measure of our cleaned samples (rescaling it so that it
corresponds to the results of DeepXi), we can see that our work does not measure
up to the current state-of-the-art.

\begin{center}
\begin{tabular}{lr}
\textbf{Method} & PESQ\\
\hline
Original & 1.97\\
Apriori SNR & 2.22\\
GAN & 2.16\\
MSE-GAN & 2.53\\
DeepXi & 2.95\\
\textbf{ConvLSTM} (our) & \textbf{2.09}\\
\end{tabular}
\end{center}

The row marked \emph{Apriori SNR} \cite{543199} is a paper from 1996 that does not use
neural networks at all, whereas both of the other approaches use GANs and custom
loss measures \cite{pascual2017segan, 8462068} - as opposed to our approach,
where we've used a convolutional network without an adversary, and the common
MSE loss.

Unfortunately, while the review part of DeepXi contained more than these works,
they did not include the results of the PESQ measure.

\section{Conclusion}
\label{sec:org08881c5}
Overall, we can conclude that while this approach is not untenable, it does not
reach the standards of the current state-of-the-art. There are the obvious
hyper-parameters to optimize (number or size of layers, changing the type of the
LSTM used), or using epochs or larger datasets, but perhaps using a different
loss criterium - as demonstrated the papers our work has been compared against -
might also be a useful approach to try.


\bibliographystyle{plain}
\bibliography{bibliography}
\end{document}